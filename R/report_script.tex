% Options for packages loaded elsewhere
\PassOptionsToPackage{unicode}{hyperref}
\PassOptionsToPackage{hyphens}{url}
%
\documentclass[
]{article}
\usepackage{amsmath,amssymb}
\usepackage{lmodern}
\usepackage{iftex}
\ifPDFTeX
  \usepackage[T1]{fontenc}
  \usepackage[utf8]{inputenc}
  \usepackage{textcomp} % provide euro and other symbols
\else % if luatex or xetex
  \usepackage{unicode-math}
  \defaultfontfeatures{Scale=MatchLowercase}
  \defaultfontfeatures[\rmfamily]{Ligatures=TeX,Scale=1}
\fi
% Use upquote if available, for straight quotes in verbatim environments
\IfFileExists{upquote.sty}{\usepackage{upquote}}{}
\IfFileExists{microtype.sty}{% use microtype if available
  \usepackage[]{microtype}
  \UseMicrotypeSet[protrusion]{basicmath} % disable protrusion for tt fonts
}{}
\makeatletter
\@ifundefined{KOMAClassName}{% if non-KOMA class
  \IfFileExists{parskip.sty}{%
    \usepackage{parskip}
  }{% else
    \setlength{\parindent}{0pt}
    \setlength{\parskip}{6pt plus 2pt minus 1pt}}
}{% if KOMA class
  \KOMAoptions{parskip=half}}
\makeatother
\usepackage{xcolor}
\IfFileExists{xurl.sty}{\usepackage{xurl}}{} % add URL line breaks if available
\IfFileExists{bookmark.sty}{\usepackage{bookmark}}{\usepackage{hyperref}}
\hypersetup{
  pdftitle={Structural Equation Modeling},
  pdfauthor={Julian Ibarguen},
  hidelinks,
  pdfcreator={LaTeX via pandoc}}
\urlstyle{same} % disable monospaced font for URLs
\usepackage[margin=1in]{geometry}
\usepackage{graphicx}
\makeatletter
\def\maxwidth{\ifdim\Gin@nat@width>\linewidth\linewidth\else\Gin@nat@width\fi}
\def\maxheight{\ifdim\Gin@nat@height>\textheight\textheight\else\Gin@nat@height\fi}
\makeatother
% Scale images if necessary, so that they will not overflow the page
% margins by default, and it is still possible to overwrite the defaults
% using explicit options in \includegraphics[width, height, ...]{}
\setkeys{Gin}{width=\maxwidth,height=\maxheight,keepaspectratio}
% Set default figure placement to htbp
\makeatletter
\def\fps@figure{htbp}
\makeatother
\setlength{\emergencystretch}{3em} % prevent overfull lines
\providecommand{\tightlist}{%
  \setlength{\itemsep}{0pt}\setlength{\parskip}{0pt}}
\setcounter{secnumdepth}{5}
\usepackage{booktabs}
\usepackage{makecell}
\usepackage{colortbl}
\usepackage{wrapfig}
\usepackage{subfig}
\usepackage{lscape}
\usepackage{natbib}
\newcommand{\blandscape}{\begin{landscape}}
\newcommand{\elandscape}{\end{landscape}}
\usepackage{subfig}
\ifLuaTeX
  \usepackage{selnolig}  % disable illegal ligatures
\fi

\title{Structural Equation Modeling}
\author{Julian Ibarguen}
\date{25/08/2022}

\begin{document}
\maketitle

{
\setcounter{tocdepth}{3}
\tableofcontents
}
\hypertarget{introduction}{%
\subsection{Introduction}\label{introduction}}

We aim to fit a Structural Equation Model (SEM) following the exercise
proposed by Petri Nokelainen from the Research Centre for Vocational
Education, University of Tampere, Finland. Data and documentation for
this exercise can be found
\href{https://www.scribd.com/document/222003310/Sem-Exercise-v2-5}{\textcolor{blue}{here}}.

We used the proposed data set with 447 rows and 27 columns. The
respondents in both samples are staff members of Finnish polytechnic
institute for higher education. The original measurement instrument
(Growth-oriented Atmosphere Questionnaire, GOAQ) has 13 factors and
92-items (Ruohotie, 1996; Ruohotie, Nokelainen \& Tirri, 2002), but for
the purposes of this exercise Nokelainen selected 13 factors and 27
items, all consisting of a five--point Likert scale from 1 (totally
disagree) to 5 (totally agree) was applied (Nokelainen, n.d.)

\hypertarget{evaluation-of-assumptions}{%
\subsection{Evaluation of assumptions}\label{evaluation-of-assumptions}}

\hypertarget{univariate-distribution}{%
\subsubsection{Univariate distribution}\label{univariate-distribution}}

Although data is ordinal, thus non-normal in nature, we check for
normality assumption. We will apply z re-scaling to see how they
approximate to mean 0. Will also check kurtosis and skewnewss, and plot
the histograms. We reject applying Saphiro test or other normality test,
as being only 5 points scale ordinal is unlikely will provide any
meaningful result.

Given a normally distribute variable would have a skewness of 0 and
kurtosis of 3. For this purpose we calculated a synthetic measure for
skewness and kurtosis as the average between the skewness and excess
kurtosis (\(kurtosis - 3\)). Under this synthetic measure, a perfect
normal distribution would have a value of 0. For our purpose, we select
those variables whose deviance from the normal distribution is lower
than 0.4, according to the synthetic measure create. The selected
variables were further confirmed with the histogram.

Observing table 1 and figure 1, we assumed normality for the following
variables: v2, v3, v4, v15, v31, v33. The remaining variables were
assumed non-normally distributed

\begin{table}[!h]

\caption{\label{tab:norm_descr}Descriptive statistics to assess approximation to normal distribution}
\centering
\begin{tabular}[t]{lcccc}
\toprule
\textbf{variable} & \textbf{skewness} & \textbf{kurtosis} & \textbf{kurtosis\_excess} & \textbf{avg\_skew\_kurt}\\
\midrule
\cellcolor{gray!6}{v1} & \cellcolor{gray!6}{-0.072} & \cellcolor{gray!6}{2.239} & \cellcolor{gray!6}{-0.761} & \cellcolor{gray!6}{0.417}\\
v2 & 0.082 & 2.350 & -0.650 & 0.366\\
\cellcolor{gray!6}{v3} & \cellcolor{gray!6}{-0.014} & \cellcolor{gray!6}{2.318} & \cellcolor{gray!6}{-0.682} & \cellcolor{gray!6}{0.348}\\
v4 & -0.203 & 2.592 & -0.408 & 0.306\\
\cellcolor{gray!6}{v5} & \cellcolor{gray!6}{-0.812} & \cellcolor{gray!6}{2.698} & \cellcolor{gray!6}{-0.302} & \cellcolor{gray!6}{0.557}\\
\addlinespace
v6 & -0.743 & 2.521 & -0.479 & 0.611\\
\cellcolor{gray!6}{v7} & \cellcolor{gray!6}{-0.496} & \cellcolor{gray!6}{2.600} & \cellcolor{gray!6}{-0.400} & \cellcolor{gray!6}{0.448}\\
v8 & -0.419 & 2.368 & -0.632 & 0.526\\
\cellcolor{gray!6}{v9} & \cellcolor{gray!6}{-0.382} & \cellcolor{gray!6}{2.454} & \cellcolor{gray!6}{-0.546} & \cellcolor{gray!6}{0.464}\\
v12 & 0.607 & 2.431 & -0.569 & 0.588\\
\addlinespace
\cellcolor{gray!6}{v13} & \cellcolor{gray!6}{0.555} & \cellcolor{gray!6}{2.531} & \cellcolor{gray!6}{-0.469} & \cellcolor{gray!6}{0.512}\\
v14 & -0.271 & 2.333 & -0.667 & 0.469\\
\cellcolor{gray!6}{v15} & \cellcolor{gray!6}{0.022} & \cellcolor{gray!6}{2.291} & \cellcolor{gray!6}{-0.709} & \cellcolor{gray!6}{0.366}\\
v16 & -0.578 & 2.295 & -0.705 & 0.641\\
\cellcolor{gray!6}{v17} & \cellcolor{gray!6}{-0.372} & \cellcolor{gray!6}{2.243} & \cellcolor{gray!6}{-0.757} & \cellcolor{gray!6}{0.565}\\
\addlinespace
v18 & -0.164 & 1.980 & -1.020 & 0.592\\
\cellcolor{gray!6}{v29} & \cellcolor{gray!6}{0.042} & \cellcolor{gray!6}{2.084} & \cellcolor{gray!6}{-0.916} & \cellcolor{gray!6}{0.479}\\
v30 & -0.107 & 2.141 & -0.859 & 0.483\\
\cellcolor{gray!6}{v31} & \cellcolor{gray!6}{-0.198} & \cellcolor{gray!6}{2.447} & \cellcolor{gray!6}{-0.553} & \cellcolor{gray!6}{0.375}\\
v33 & -0.610 & 2.876 & -0.124 & 0.367\\
\addlinespace
\cellcolor{gray!6}{v34} & \cellcolor{gray!6}{-0.806} & \cellcolor{gray!6}{3.277} & \cellcolor{gray!6}{0.277} & \cellcolor{gray!6}{0.542}\\
v42r & -0.026 & 2.137 & -0.863 & 0.444\\
\cellcolor{gray!6}{v43} & \cellcolor{gray!6}{-0.716} & \cellcolor{gray!6}{2.875} & \cellcolor{gray!6}{-0.125} & \cellcolor{gray!6}{0.420}\\
v44 & -0.341 & 2.133 & -0.867 & 0.604\\
\cellcolor{gray!6}{v45} & \cellcolor{gray!6}{0.522} & \cellcolor{gray!6}{2.299} & \cellcolor{gray!6}{-0.701} & \cellcolor{gray!6}{0.612}\\
\addlinespace
v46 & 0.558 & 2.384 & -0.616 & 0.587\\
\cellcolor{gray!6}{v47} & \cellcolor{gray!6}{0.596} & \cellcolor{gray!6}{2.271} & \cellcolor{gray!6}{-0.729} & \cellcolor{gray!6}{0.663}\\
\bottomrule
\end{tabular}
\end{table}

\begin{figure}
\centering
\includegraphics{report_script_files/figure-latex/histogram-1.pdf}
\caption{Histograms}
\end{figure}

\hypertarget{bivariate-distribution}{%
\subsubsection{Bivariate distribution}\label{bivariate-distribution}}

Based on the univariate analysis, we select three different variable
combinations: 1) both are assumed normally distributed; 2) one is
assumed normally distributed, but the other one no; 3) neither fo them
is assumed normally distributed.

In figure 2, we can observe the linear relationship between the 3
different combination. When we assume normality in both variables, we
observe linear relationship, which decreases for the second combination
(one variable assumed normal and the other no), and the relationship
disappear for the third combination

\begin{figure}

{\centering \subfloat[Two variables assumed normally distributed\label{fig:bivariates-1}]{\includegraphics[width=0.5\linewidth]{report_script_files/figure-latex/bivariates-1} }\subfloat[One variable assumed normally distributed\label{fig:bivariates-2}]{\includegraphics[width=0.5\linewidth]{report_script_files/figure-latex/bivariates-2} }\newline\subfloat[No variable assumed normally distributed\label{fig:bivariates-3}]{\includegraphics[width=0.5\linewidth]{report_script_files/figure-latex/bivariates-3} }

}

\caption{Bivariate distributions}\label{fig:bivariates}
\end{figure}

\hypertarget{correlation-matrix}{%
\subsubsection{Correlation matrix}\label{correlation-matrix}}

We observe now the correlation matrix between all 27 variables. For SEM
and other factorial type of analysis, the correlation coefficient should
be between \(\pm 0.3\) and \(\pm 0.8\). Too low correlations indicate
weak inter-item dependency, too high correlations might indicate
multicollinearity (Nokelainen, n.d.).

The highest correlation coefficient (\(r\)) in absolute terms was betwee
v18, v17 with \$r = \$0.8344792 and \(R^2\) =
\texttt{r}max\_r2\texttt{.\ The\ lowest\ correlation\ coefficient\ (\$r\$)\ was\ between\ v45,\ v7\ with\ \$r\ =\ \$-0.3372552\ and\ \$R\^{}2\$\ =}r
\texttt{min\_r2}. Therefore, we cna conlcude that our variables are
suitable for a SEM model

\begin{figure}
\centering
\includegraphics{report_script_files/figure-latex/cor_plot-1.pdf}
\caption{Correlation matrix}
\end{figure}

\hypertarget{path-analysis}{%
\subsection{Path analysis}\label{path-analysis}}

\hypertarget{model-1}{%
\subsubsection{Model 1}\label{model-1}}

To perform the path analysis we make use of the 13 factors available in
our data. The data consists of the thirteen growth-oriented atmosphere
factors. The sample is the same as in data1, 447 staff members of
Finnish polytechnic institute for higher education. The sample was
collected in 2000.

Over this data we are interested on knwoing the determinant for
\emph{Valuation of the work}. We fit the following model:

\[ta\_val7 = \beta_0 + \beta_1 kj\_enc1 + \beta_2 op\_rew3 + \beta_3 tk\_inv5 + \beta_4ts\_cla + \varepsilon_{ta\_val7}\]

where:

\begin{table}[!h]

\caption{\label{tab:var_names}Labels for factor variables}
\centering
\begin{tabular}[t]{lc}
\toprule
\textbf{variable\_label} & \textbf{variable\_name}\\
\midrule
\cellcolor{gray!6}{1. Encouraging leadership} & \cellcolor{gray!6}{kj\_enc1}\\
2. Strategic leadership & sj\_str2\\
\cellcolor{gray!6}{3. Know-how rewarding} & \cellcolor{gray!6}{op\_rew3}\\
4. Know-how developing & ok\_dev4\\
\cellcolor{gray!6}{5. Incentive value of the work} & \cellcolor{gray!6}{tk\_inv5}\\
\addlinespace
6. Clarity of the work & ts\_cla6\\
\cellcolor{gray!6}{7. Valuation of the work} & \cellcolor{gray!6}{ta\_val7}\\
8. Relationship-based learning & vk\_rel8\\
\cellcolor{gray!6}{9. Team spirit} & \cellcolor{gray!6}{rh\_tes9}\\
11. Psychical stress of the work & tr\_psy11\\
\addlinespace
\cellcolor{gray!6}{12. Build-up of work requirements} & \cellcolor{gray!6}{tv\_bui12}\\
13. Commitment to work and organization & si\_com13\\
\bottomrule
\end{tabular}
\end{table}

The summary of the model was the following:

\begin{verbatim}
## lavaan 0.6-12 ended normally after 16 iterations
## 
##   Estimator                                         ML
##   Optimization method                           NLMINB
##   Number of model parameters                         6
## 
##                                                   Used       Total
##   Number of observations                           443         447
## 
## Model Test User Model:
##                                                       
##   Test statistic                                 0.000
##   Degrees of freedom                                 0
## 
## Parameter Estimates:
## 
##   Standard errors                             Standard
##   Information                                 Expected
##   Information saturated (h1) model          Structured
## 
## Regressions:
##                    Estimate  Std.Err  z-value  P(>|z|)   Std.lv  Std.all
##   ta_val7 ~                                                             
##     kj_enc1           0.381    0.047    8.086    0.000    0.381    0.446
##     op_rew3           0.087    0.042    2.056    0.040    0.087    0.092
##     tk_inv5           0.228    0.040    5.697    0.000    0.228    0.221
##     ts_cla6           0.080    0.043    1.864    0.062    0.080    0.092
## 
## Intercepts:
##                    Estimate  Std.Err  z-value  P(>|z|)   Std.lv  Std.all
##    .ta_val7           0.959    0.140    6.842    0.000    0.959    1.093
## 
## Variances:
##                    Estimate  Std.Err  z-value  P(>|z|)   Std.lv  Std.all
##    .ta_val7           0.365    0.025   14.883    0.000    0.365    0.473
## 
## R-Square:
##                    Estimate
##     ta_val7           0.527
\end{verbatim}

\begin{figure}

{\centering \includegraphics[width=0.6\linewidth]{report_script_files/figure-latex/path_plot-1} 

}

\caption{Model path}\label{fig:path_plot}
\end{figure}

From the results of the model we obtain the following conclusions:

\begin{itemize}
\item
  How much dependent variable variance the four independent variables
  predict? \(R^2 = 0.527\) .
\item
  Order the IVs in the following rows (best predictor comes first):

  \begin{enumerate}
  \def\labelenumi{\arabic{enumi}.}
  \tightlist
  \item
    The first (strongest) predictor for Valuation of the work is
    \emph{1. Encouraging leadership} \(r = 0.45\)
  \item
    The second predictor for Valuation of the work is \emph{5. Incentive
    value of the work} \(r = 0.22\)
  \item
    The third predictor for Valuation of the work is \emph{3. Know-how
    rewarding} \(r = 0.09\)
  \item
    The fourth predictor for Valuation of the work is \emph{6. Clarity
    of the work} \(r = 0.09\)
  \end{enumerate}
\item
  Select Unstandardized estimates and complete the following sentences:

  \begin{itemize}
  \tightlist
  \item
    When \emph{Encouraging leadership} goes up by 1, \emph{Valuation of
    the work} goes up by 0.38.
  \item
    When \emph{Know-how rewarding} goes up by 1, \emph{Valuation of the
    work} goes up 0.09.
  \end{itemize}
\end{itemize}

\hypertarget{model-2}{%
\subsubsection{Model 2}\label{model-2}}

We modify the model 1 to account for an indirect effect from
\emph{Know-how rewarding} (op\_rew3) via \emph{Incentive value of of the
work} (tk\_inv5) to \emph{Valuation of the work} (ta\_val7).

\[ta\_val7 = \beta_0 + \beta_1 kj\_enc1 + \beta_2 op\_rew3 + \beta_3ts\_cla + \beta_4 op\_rew3 * tk_inv5 + \varepsilon_{ta\_val7}\]

By accounting for the mediation effect, we obtain the following result:

\begin{verbatim}
## lavaan 0.6-12 ended normally after 7 iterations
## 
##   Estimator                                         ML
##   Optimization method                           NLMINB
##   Number of model parameters                         7
##   Number of equality constraints                     2
## 
##                                                   Used       Total
##   Number of observations                           443         447
## 
## Model Test User Model:
##                                                       
##   Test statistic                                88.828
##   Degrees of freedom                                 4
##   P-value (Chi-square)                           0.000
## 
## Parameter Estimates:
## 
##   Standard errors                             Standard
##   Information                                 Expected
##   Information saturated (h1) model          Structured
## 
## Regressions:
##                    Estimate  Std.Err  z-value  P(>|z|)   Std.lv  Std.all
##   ta_val7 ~                                                             
##     kj_enc1    (c)    0.186    0.012   15.498    0.000    0.186    0.223
##     op_rew3    (c)    0.186    0.012   15.498    0.000    0.186    0.201
##     ts_cla6    (c)    0.186    0.012   15.498    0.000    0.186    0.218
##   tk_inv5 ~                                                             
##     op_rew3    (a)    0.378    0.040    9.478    0.000    0.378    0.411
##   ta_val7 ~                                                             
##     tk_inv5    (b)    0.254    0.037    6.926    0.000    0.254    0.253
## 
## Variances:
##                    Estimate  Std.Err  z-value  P(>|z|)   Std.lv  Std.all
##    .ta_val7           0.380    0.026   14.883    0.000    0.380    0.518
##    .tk_inv5           0.601    0.040   14.883    0.000    0.601    0.831
## 
## R-Square:
##                    Estimate
##     ta_val7           0.482
##     tk_inv5           0.169
## 
## Defined Parameters:
##                    Estimate  Std.Err  z-value  P(>|z|)   Std.lv  Std.all
##     ab                0.096    0.017    5.592    0.000    0.096    0.104
##     total             0.282    0.018   15.663    0.000    0.282    0.327
\end{verbatim}

\begin{figure}

{\centering \includegraphics[width=0.6\linewidth]{report_script_files/figure-latex/path_plot_med-1} 

}

\caption{Model path}\label{fig:path_plot_med}
\end{figure}

And we can extract the following conclusions:

\begin{itemize}
\tightlist
\item
  How much DV‟s variance the four IV‟s predict? Model 1 \(R^2 = 0.527\).
  Compared to Model 2 with mediation 0.65 \%.
\item
  How does the indirect path affect the regression model? controlling
  for the mediation effect have improved the effect of Clarity of the
  work (ts\_cla6) and \emph{Know-how rewarding} (op\_rew3). On the
  contrary have decreased the effect of \emph{Encouraging leadership}
  (kj\_enc1). However, the new model seems to explain better the
  \emph{Valuation of the work} (ta\_val7),a s per \(R2\) metric.
\end{itemize}

\end{document}
